\documentclass[12pt]{article}
\usepackage{amsmath}
\usepackage{amsfonts}
\usepackage{geometry}
\usepackage[utf8]{inputenc}
\usepackage[T1]{fontenc}
\usepackage[macedonian]{babel}

\begin{document}
\begin{titlepage}
    \begin{center}
        \Large{\textbf{
                \begin{tabular}{c c}
                    Марија Оровчанец     & Билјана Крстеска \\
                    Весна Манова-Ерковиќ & Ѓорѓи Маркоски
                \end{tabular}
            }}
        \vfill
        \LARGE{\textbf{ЗБИРКА РЕШЕНИ ЗАДАЧИ ПО МАТЕМАТИЧКА АНАЛИЗА I\\ (ВТОР ДЕЛ)}}
        \vfill
        \Large{\textbf{Скопје, 2015}}
        \vfill
    \end{center}
\end{titlepage}
\begin{flushleft}
    \section*{Рецензенти}
    \textbf{Проф. д-р Никола Пандески}\\
    Редовен професор на ПМФ, Скопје\\
    \textbf{Проф. д-р Никита Шекутковски}\\
    Редовен професор на ПМФ, Скопје\\
\end{flushleft}
\vfill
{\fontfamily{cmss}\selectfont
    Со одлука број 07-236/7 од 30.9.2004 година, на
    Наставно-научниот совет на Природно-математичкиот
    факултет во Скопје се одобрува печатење на оваа книга како
    учебно помагало.
}
\clearpage
\begin{center}
    \large{\textbf{ПРЕДГОВОР}}
\end{center}
\par
\large
Oваа збирка задачи пред се е наменета за студентите на
студиите по математика. Но, сметаме дека збиркава ќе биде
интересна и за студентите од техничките факултети и сите оние кои
сакаат да се запознаат со основните поими од математичката
анализа.\par
Збиркава содржи пет глави. На почетокот на секоја од нив
дадени се дефиниции и основни својства на поимите кои се
разгледуваат во таа глава. Потоа следуваат формулациите и
решенијата задачите. Доказите на теоремите и целата потребна
теоретска подготовка може да се најдат во учебниците [1] и [2].\par
Сите задачи се детално решени. Пожелно е читателот да се
обиде да ја реши задачата, а ако не успее, дури тогаш да го погледне
решението. За да може успешно да се следи материјалот во оваа
збирка, потребно е читателот да има добри предзнаења од
материјалот кој се изучува во средното образование.\par
На крајот збиркава содржи задачи од писмените испити по
математичка анализа 1 на студиите по математика на Природноматематичкиот факултет во Скопје.\par
Им се заблагодаруваме на рецензентите кои дадоа корисни
забелешки и предлози за подобрување на оваа збирка задачи. Сите
забелешки од читателите ќе бидат добродојдени за подобрување на
текстот на оваа книга во иднина.
\vfill
\clearpage
\newpage
\
\newpage
\end{document}

